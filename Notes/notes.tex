\documentclass[10pt]{article}


% Tous les packages prédéfinis
\usepackage{introLatex}
\usepackage{headfootLatex}
\usepackage{shortcutLatex}
\usepackage{envLatex}
\usepackage{booktabs}
\usepackage{algorithm}
%\usepackage{algpseudocode}

\graphicspath{{logos/}{figures/}}




\makeatletter
\def\hlinewd#1{%
\noalign{\ifnum0=`}\fi\hrule \@height #1 %
\futurelet\reserved@a\@xhline}
\makeatother


\begin{document}


% Titre du document
\vspace*{-22pt}
\begin{center}
\textbf{\Large Etude mathématique et numérique \\ du groupe de renormalisation non perturbatif}\\
\vspace*{4pt}
Gaétan Facchinetti \\
{\small 27 février - 28 juillet 2017\\
\vspace*{5pt}
\textit{Laboratoire de Physique Théorique de la Matièe Condensée},\\
\textit{Université Paris-Saclay}, \textit{Ecole Normale Supérieure de Cachan}, \\
\textit{Ecole Nationale Supérieure des Techniques Avancées}}\\
\end{center}


\begin{center}
\rule{10cm}{1pt}
\end{center}

%\vspace*{4pt}

\begin{multicols}{2}

\section{Le groupe de renormalisation}

\subsection{Introduction}

On considère un système physique en dimension $d$ et symétrique par le groupe de rotation $O(N)$. Il peut alors être décrit par un champ $\varphiv$ de $\R^d$ dans $ \R^N$. Toutes les informations que l'on peut souhaiter avoir sur ce système sont, dès lors, contenu dans sa fonction de partition dont l'expression purement formelle est 

\begin{equation}
\Zc[\hv] = \int \Dr \varphiv e^{-H[\varphiv] + \int_{\R^d} \hv \varphiv} 
\end{equation} 

Où l'intégrale est une intégrale fonctionnelle, sur l'ensemble des champs $\varphiv$ possibles. $\hv$ représente une exitation extérieure, $H[\varphiv]$ est le hamiltonien du système (prenant en compte la dépendance en température).\\

Cependant, les intégrales fonctionnelles ne sont que des objets formels sur lesquels il est impossible de réaliser une résolution analytique ou numérique directelemnt. L'objectif du groupe de renormalisation est ainsi de parvenir à calculer, sous certaines approximations, des grandeurs ratachées à cette fonction de partition, contenant l'information souhaitée. 

\pagebreak

\appendix

\section{Rappels de calculs}

\subsection{Derivation fonctionnelle}


\textbf{Définition}\\
Soit $U$ et $V$ deux espaces de Banach. Soit $F$ une fonctionnelle de $U$ dans $V$. 
Soit $f \in U$. On appelle, si elle existe, dérivée (au sens de Fréchet) de la fonctionelle $F$ prise en $f$, l'application linéaire de $\mathcal{L}(U,V)$, notée $D_fF$ telle que, pour $\eps>0$, 

\begin{equation}
\forall h \in U \quad \lim_{ \eps \rightarrow 0 } \frac{\| F[f+\eps h] - F[f] - \eps D_fF.h\|_V}{\eps}  = 0
\end{equation} 


Dans notre étude nous recontrerons essentiellement $V = (\R, |.|)$. On utilisera alors aussi les crochets de dualités, qui permettent de réécrire l'action d'un élément $T$ de $U^*$ sur $U$ (et de les identifier) : 

\begin{equation}
  \forall h \in U \quad T.h = \left< T, h \right>_{U^*, U}
\end{equation}

Ainsi, si la dérivée au sens de Fréchet de $F$ existe (donc si $F$ est Frechet différentlable en $f$) alors on note $\derd{F}{f}$ l'application de $U^*$ définie par 

\begin{equation}
  \forall h \in U \quad D_fF.h = \left< \derd{F}{f}, h \right>_{U^*, U}
\end{equation}

\vspace*{11pt}



\subsection{Transformées de Fourier}

On rappelle ici les notations utilisée pour définir les transformées de Fourier. Pour cela on considère $f$ une application de $L^2(\R^d, \R^N)$. On définit alors la transformée de Fourier de $f$, notée $\hat{f}$ par, 

\begin{equation}
  \forall q \in \R^d \quad \hat{f}(q) = \frac{1}{\sqrt{(2\pi)^d}}\int_{\R^d} f(x) e^{-iqx}\dd x
\end{equation}

Nous avons alors la relation inverse,

\begin{equation}
  \forall x \in \R^d \quad f(x) = \frac{1}{\sqrt{(2\pi)^d}}\int_{\R^d} \hat{f}(q) e^{iqx}\dd q
\end{equation}

\textbf{Remarque 1} \\
Dans le cas ou l'on a des fonctions définies non pas sur $\R^d$ mais sur un domaine $\Omega \subset \R^d$ fini alors nous aurons les  mêmes propriétés avec la relation "d'équivalence" : 

\begin{equation}
  \frac{1}{(2\pi)^d} \int_{\R^d} \dd^d \qv \overset{\text{$\Omega$ fini}}{\quad \longrightarrow \quad } \frac{1}{\Omega} \sum_{\qv}
\end{equation}


\textbf{Remarque 2} \\
Dans le cadre où $f \in \Sr'(\R^d, \R^N)$ (espace des distributions tempérées) alors on étend la notion de transformée de Fourier $\hat{f}$ de $f$ à l'aide du crochet de dualité, 
\begin{equation}
  \forall \phi \in \Sr(\R^d, \R^N) \quad \left< \hat{f}, \phi \right>_{\Sr', \Sr} = \left< f, \hat{\phi} \right>_{\Sr', \Sr}
\end{equation}

\vspace*{11pt}







\vspace*{11pt}

\subsection{TF et dérivées fonctionnelles}


On expose ici la justification d'un résultat très souvent utilisé dans la dérivation des équations du groupe de renormalisation non perturbatif.\\ 

%\noident
\textbf{Proposition}

Soit $F$ une fonctionelle de $U$ dans $(\R,|.|)$ et $f \in U$.
Soit $x \in \R^d$. On suppose que $U = L^2(\R^d, \R^N)$. 
Alors, 
\begin{equation}
  \derd{F}{f(x)} = \frac{1}{\sqrt{(2\pi)^d}} \int_{\R^d} \derd{F}{\hat{f}(-q)} e^{iqx} \dd^d q
\end{equation} 


\vspace*{11pt}
%\noindent
\textbf{Demonstration}

Par la règle de la chaine de la dérivation de Fréchet nous pouvons écrire, 
\begin{equation}
  \forall h \in U \quad D_fF.h = D_{\hat{f}}F. D_f \hat{f}.h
\end{equation}


\commentout{
Ainsi, ceci nous donne donc, pour $x \in \R^d$
\begin{equation}
  \derd{F}{f(x)} = \int_{\R^d} \derd{\hat{f}(q)}{f(x)}\derd{F}{\hat{f}(q)} \dd^d q
\end{equation}
}

Cependant, $\hat{f}$ est une fonctionnelle de $f$ (par définition de la TF) de $U$ dans $U$ qui est linéaire en $f$. Soit $\eps >0$, 
\begin{equation}
  \forall h \in U \quad \hat{f}[f + \eps h] - \hat{f}[f] = \eps \hat{h} 
\end{equation} 

Il vient directement, par définition de la dérivation au sens de Frechet, $D_f\hat{f} .h = \hat{h}$.  
Ainsi,  $D_f F .h = D_{\hat{f}}F.\hat{h}$.  On peut alors écrire, 
\begin{align}
  \forall h \in U  \quad D_f F .h & = \int_{\R^d} \derd{F}{f(q)} \int_{\R^d} h(x) e^{-iqx} \dd^d x \dd^d q \\
  \forall h \in U  \quad D_f F .h & = \int_{\R^d} \int_{\R^d} \derd{F}{f(q)} e^{-iqx} h(x) \dd^d q \dd^d x \\
  \forall h \in U  \quad D_f F .h & = \int_{\R^d} \int_{\R^d} \derd{F}{f(-q)} e^{iqx} \dd^d q \text{ } h(x) \dd^d x
\end{align}

Ce qui permet de conclure la démonstration.

\vspace*{11pt}

\end{multicols}


\pagebreak

\section{Opérateur à noyaux}

Nous détaillons dans cette sections quelques propriétés élémentaires mais très utiles dans notre études des opérateurs à noyaux. Après une définition rigoureuse de ces opérateurs et de leur trace nous utiliserons une description plus formelle de ce que peut être leur inverse.\\



\subsection{Definition}

Soit $S$ un endomorphisme de . On appelle $S$ un operateur a noyaux, une application de $L^2(\R^d, \R^N)$, telle que pour tout $(i,j) \in \bbrac{1,N}^2$, il existe une application $A_{i,j} \in L^2(\R^d\times\R^d, \R)$ telle que,  pour tout $f \in L^2(\R^d, \R^N)$, et $x \in \R^d$ 
 \begin{equation}
  S[f]_i(x) = \int_{\R^d} \sum_{j=1}^{N} A_{i,j}(x,y)f_j(y) \dd^d y
 \end{equation}

 La matrice $A : (x,y) \rightarrow ((A_{i,j}(x,y)))_{i,j}$ est appellée noyaux de $S$. \\
 On identifiera alors dans les notations $S$ et $A$ indépendemment. \\

\textbf{Proposition}\\
Par Cauchy-Schwartz cette définition à un sens, $S$ est bien définie. \\
De plus $S$ et est un endomorphisme de $L^2(\R^d, \R^N)$. \\

\textbf{Démonstration}\\
Soit $f\in L^2(\R^d, \R^N)$.\\

\begin{equation}
   \int_{\R^d} \left| S[f]_i (\xv) \right|^{2} \dd^d \xv \le \int_{\R^d} \int_{\R^d} \sum_j 
\end{equation}


\vspace*{11pt}

\subsection{TF d'un opérateur à noyaux}

Nous avons vu que $S$ définie comme précédemment était un endomorphisme de $L^2(\R^d, \R^N)$. Ainsi il est possible d'en définir la transformée de Fourier. \\

\textbf{Proposition}\\
Pour $S$ opérateur à noyaux défini comme ci-dessus, nous avons,
\begin{equation}
  \hat{S}[f]_i(\qv) = \int_{\R^d} \sum_{j=1}^{N} \hat{A}_{i,j}(\qv, -\qv') \hat{f}_j(\qv')
\end{equation}



\textbf{Démonstration}\\
Nous développons le calcul de la transformée de Fourier. Soit $\qv \in \R^d$,
\begin{align}
 \hat{S}[f]_i(\qv)  = & \int_{\R^d} \int_{\R^d} \sum_{j=1}^{N} A_{i,j}(\xv,\yv)f_j(\yv) e^{-i\qv\xv} \dd^d \yv \dd^d \xv \\
 \hat{S}[f]_i(\qv)  = & \int_{\R^{3d}} \sum_{j=1}^{N} A_{i,j}(\xv,\yv) \hat{f}_j(\qv') e^{i\qv'\yv} e^{- i\qv\xv}  \dd^d \qv' \dd^d \yv \dd^d \xv \\
 \hat{S}[f]_i(\qv)  = &    \sum_{j=1}^{N} \int_{\R^d} \hat{f}_j(\qv') \left\{ \int_{\R^{2d}} A_{i,j}(\xv,\yv)  e^{i\qv'\yv} e^{- i\qv\xv}  \dd^d \yv \dd^d \xv   \right\} \dd^d \qv'\\
\hat{S}[f]_i(\qv)  = &    \sum_{j=1}^{N} \int_{\R^d} \hat{f}_j(\qv') \hat{A}_{i,j}(\qv, -\qv') \dd^d \qv'\\
\end{align}

\vspace*{11pt}

\subsection{Produit de Volterra}

Considérons deux opérateurs à noyaux, $S$ et $T$ de noyaux respectifs $A$ et $B$. \\
La composition de ces deux opérateur s'écrit comme suit,
\begin{equation}
  \forall \xv \in \R^d \quad S \circ T [f]_i (\xv) = \int_{\R^d}  \sum_{j=1}^{N} \left\{ \int_{\R^d} \sum_{j=k}^{N}A_{i,k}(\xv, \zv)B_{k,j}(\zv, \yv)  \dd^d \yv \right\} f_j(\yv)  \dd^d \yv
\end{equation}

On notera alors aussi
\begin{equation}
    \forall (\xv, \yv) \in (\R^d)^2 \quad (A \circ B)_{i,j} (\xv, \yv) = \int_{\R^d} \sum_{j=k}^{N}A_{i,k}(\xv, \zv)B_{k,j}(\zv, \yv)  \dd^d \yv 
\end{equation}

\textbf{Proposition}\\
Par transformée de Fourier, par un calcul anaolgue à celui fait pour un simple noyaux, il vient,
\begin{equation}
\forall (\qv, \qv') \in (\R^d)^2 \quad\widehat{(A \circ B)}_{i,j}(\qv, \qv') =\int_{\R^d} \sum_{j=k}^{N}\hat{A}_{i,k}(\qv, \qv'')\hat{B}_{k,j}(-\qv'', \qv')  \dd^d \qv'' 
\end{equation}



\vspace*{11pt}

\subsection{Trace d'un opérateur à noyaux}

\textbf{Definition}\\
On définit la trace d'un opérateur $S$ de noyaux $A$ par
\begin{equation}
  \text{Tr} A = \sum_{i=1}^{N} \int_{\R^d} A_{i,i}(\xv,\xv) \dd^d \xv 
\end{equation}


\textbf{Proposition}\\
Nous pouvons de manière similaire à ce qui a été fait pour démontrer l'expression de la transformée de Fourier d'un opérateur à noyaux en (...) dériver la relation sur la trace de la transformée de Fourier
\begin{equation}
  \text{Tr} \hat{A} = \sum_{i=1}^{N} \int_{\R^d} \hat{A}_{i,i}(\qv,-\qv) \dd^d \qv 
\end{equation}



\vspace*{11pt}



\subsection{Dérivation de l'inverse d'une matrice}

Afin de s'approprier les propriétés des opérateurs à noyaux, de par leur similarités avec les matrices nous commenceront par développer de petites propriétés matricielles que nous pourrons transposer par la suite.

\vspace*{11pt}

\textbf{Proposition} \\
Soit $ n \in \mathbb{N}$. Soit $((A_{i,j}))_{i,j}$ une matrice de $\R^{n \times n}$ telles que pour tout $(i,j) \in \bbrac{1,n}$, $A_{i,j}$ soit une application de $\mathscr{C}^1(\R,\mathbb{C}) $. On suppose que pour tout $x \in \R$, $A(x) \in  \mathbb{G}L_n(\R)$. Alors, 
\begin{equation}
  \forall x \in \R \quad \frac{dA^{-1}(x)}{dx} = - A^{-1}(x) \frac{dA(x)}{dx}A^{-1}(x)
\end{equation} 


\textbf{Demonstration}\\
Soit $x \in \R$.
Par définition de $A^{-1}$, $AA^{-1} = \text{Id}_{\R^{n\times n}}$.\\
En dérivant cette relation nous obtenons, 
\begin{equation}
  \frac{dA(x)}{dx}A^{-1}(x) + A\frac{dA^{-1}(x)}{dx} = 0
\end{equation}

Il vient directement le résultat.



\section{Formules en vrac}

\begin{equation}
   \delta(\xv - \yv) = \int_{\R^d} F(\xv, \zv) F^{-1}(\zv, \yv) \dd^d \zv
\end{equation}

\begin{equation}
   \delta_{i,j} \delta(\xv - \yv) = \int_{\R^d} \sum_{k=1}^{N} F_{i,k}(\xv, \zv) F^{-1}_{k,j}(\zv, \yv) \dd^d \zv
\end{equation}

\begin{equation}
   \delta(\qv_1 +\qv_3) = \int_{\R^d} \hat{F}(\qv_1, -\qv_2) \hat{F}^{-1}(\qv_2, \qv_3) \dd^d \qv_3
\end{equation}


\begin{equation}
   \delta_{i,j}\delta(\qv_1 +\qv_3) = \int_{\R^d} \sum_{k=1}^{N} \hat{F}_{i,k}(\qv_1, -\qv_2) \hat{F}^{-1}_{k,j}(\qv_2, \qv_3) \dd^d \qv_3
\end{equation}

\begin{equation}
   \derd{\hat{F}}{\hat{\phiv}(-\pv)}(\qv_1, \qv_4) = - \int_{\R^d} \int_{\R^d} \hat{F}^{-1}(\qv_1, -\qv_2) \derd{\hat{F}}{\hat{\phiv}(-\pv)}(\qv_2, \qv_3) \hat{F}^{-1}(-\qv_3, \qv_4) \dd^d \qv_2 \dd^d \qv_3
\end{equation}



\end{document}
